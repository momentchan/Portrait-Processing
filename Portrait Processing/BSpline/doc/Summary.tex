\documentclass[12pt,a4paper]{article}
\usepackage{amsmath, amssymb}
\usepackage{helvet}
\author{V. Srikrishnan}

\title{B Spline curve evolution and a Tangential
  Evolution Term: Implementation Notes}

\begin{document}
\maketitle
\section{Introduction}
This library is based on the work\cite{tangent-term}. Briefly, this
library provides an implementation of curve evolution equation using
B-Splines. This software collection makes heavy use of the
OpenCV\cite{opencv} library. Before describing the library itself, a
brief overview of parametric active contours is provided. This will
provide the background for understanding the code as well as
introducing the various symbols we have used in the code. The
referenes are kept to a minimum because this is not a
paper. This does not mean that the we do not acknowledge the earlier
research works! For any queries or bug-reports, please email
me at v.srikrishnan@gmail.com .
% ------------------------------------------------------
\section{Curve Evolution Equations and Active Contours}
Active contours are a well-known formulation for segmentation and
tracking tasks. Contours are planar curves; an energy functional is
defined on them. The general form of the energy functional is as 
\begin{align}\label{eq:energy}
E[C] = E_{ext}[C] + E_{int}[C]
\end{align}
where $C(p) = [x(p), y(p)]$ is the curve parameterised by $p$. The
Euler-Lagrange equations for the function leads to PDEs which are
solved numerically to obtain the solution which hopefully will segment
the object of interest. The curve evolution can be written as
\begin{equation}\label{eq:evoluton-eqn}
C_t(p,t) = \alpha(p,t) \mathbf{T}(p,t) + \beta (p, t) \mathbf{N}(p,t)
\end{equation}
Here $\mathbf{N}$ and $\mathbf{T}$ stand for the local normal and
tangent vectors respectively. This is illustrated in figure
\ref{fig:force-terms}. The important point to note is that 
\begin{itemize}
\item normal force component $\beta (p,t)$ has a role in changing the
  shape of the curve. 
\item tangent term $\alpha (p,t)$ re-parameterises the curve.
\end{itemize}
For parametric implementation of evolution equations, the curve points
bunch and space out along the curve. This causes poor segmentation and
problem of loop formation in case of B-Spline implementation. For
further details, see \cite{tangent-term}. In this work, the authors
propose a tangential term which re-parameterises the curve at every
iteration to remove the above mentioned problem. 
\emph{This does not happen with level set implementation of curve
  evolution equations}. 
\subsection{BSplines and evolution equation}
BSplines have been used extensively for implementation. There are
numerous advantages of using B-Splines for implementation. B-Spline
curve are defined as:
\begin{equation}\label{eq:bspline}
X(p) = \Sigma_{i=1}^M B(p,n) C(p) 
\end{equation}
The curve $X(p)$ is generated as the linear combination of the basis
functions $B(p,n)$ with the weights being $C(p)$. The weights are also
called control points. Notice that if the discretisation is fixed and
the basis calculated, then different curves can be generated by
different control point sets. 

We now have to implement \ref{eq:evoluton-eqn} within the B-Spline
framework. Therefore, we have to essentially update the control points
$C(p)$ at each time step. This is done in a least squares sense. Let
the curve be discretised into $N$ points. Therefore, the force moving
the curve is known at these $N$ points. Usually, $N$ is much larger
than the number of control points $M$; this is not a restrictive
assumption but is the reason that B-Splines are so popular. Let the
change in control points from time $t$ to $t+1$ be denoted by
$\partial C$ because of the external force $\mathbf{F}$ moving the
curve by $\partial X$. The change
$\partial C$ can be written following \ref{eq:bspline} as
\begin{equation}
\partial X = \mathbf F = \Sigma_{i=1}^M B(p,n)\partial C(p)
\end{equation}
Now, we have $N$ equations and $M < N$ equations which is an
overdetermined system. This is solved using least squares.
% ------------------------------------------------------
\section{Basic Datastructures}
Description of basic datastructures. From previous section, it is
clear that the basis function once calculated can be used to generate
different curves sharing the same discretisation and order. The first
major datastructure is 

struct Cv\_Basis\\
\{

int N\_b, N, K;

CvMat *Deriv1;

\}


\end{document}
